\documentclass[oneside, 12pt]{book}

\usepackage[brazil]{babel}
\usepackage[utf8]{inputenc}
\usepackage{indentfirst}
\usepackage{titling}
\usepackage[pagestyles]{titlesec}

\renewcommand{\bfseries}{}
\setcounter{tocdepth}{1}

\titleformat{\section}[display]{}{}{0pt}{\LARGE}
\newpagestyle{mystyle}
{\sethead{}{}{\thepage}}
\pagestyle{mystyle}

\newenvironment{prat}[1]
{\begin{center}
\begin{tabular}{p{0.9\textwidth}}
\hline\hline
\multicolumn{1}{c}{Convite à Prática}\\
\hline
Frequência: #1.\\
\hline
}{\\
\hline\hline
\end{tabular} 
\end{center}}

\title{Da Perfeição de Ser}
\author{Arthur Paulino}
\date{}

\begin{document}

\maketitle

\chapter*{Aviso Legal}
\thispagestyle{empty}

Ao fazer uso de qualquer informação fornecida neste livro, o leitor é inteiramente responsável por si.

\chapter*{Agradecimentos}
\thispagestyle{empty}

Sou profundamente grato aos meus pais, por terem me dado suporte mesmo nos momentos mais difíceis; a todos os seres iluminados que já existiram, que existem e que existirão e à minha companheira Larissa, que é um verdadeiro encanto.

\tableofcontents
\thispagestyle{empty}

\chapter*{Prefácio}\addcontentsline{toc}{chapter}{Prefácio}

Recomendo esta leitura para aqueles que já sentiram a angústia da pobreza espiritual e que conseguiram firmar o desejo de experienciar a Vida em sua maior plenitude. Não se trata, portanto, de um livro para ser lido às pressas. Cada frase é parte de uma argumentação estruturada para convidar o leitor à introspecção e encorajá-lo a realizar uma profunda transformação em direção à maestria de si.

A linguagem será simples e objetiva, e muitas vezes trarei a primeira pessoa. Assim o faço para manter a obra ainda mais honesta e sincera, pois aqui também estão presentes meus sentimentos de gratidão e de responsabilidade por uma humanidade mais sã, próspera e próxima de se perceber uma unidade.

\chapter*{Introdução}\addcontentsline{toc}{chapter}{Introdução}

Como podemos viver bem se há tantos motivos para reclamar? As fontes de insatisfação são praticamente incontáveis. Temos desgosto da situação social, política e econômica na qual estamos e costumamos atribuir a culpa aos nossos supostos líderes ou até mesmo à falta de opções durante as eleições. Mas já é hora de nós buscarmos um olhar mais profundo e entender o poder que temos como indivíduos.

É fato que os políticos deste período da humanidade chegam ao poder com o auxílio do capital e com o intuito servi-lo. Com a ganância das grandes corporações no poder, estamos sujeitos a um crescimento cada vez mais polarizado em direção ao benefício de poucos e à destruição dos ecossistemas deste planeta.

Precisamos entender a raiz do problema. Afinal, qual a origem do poder dessas grandes corporações mal intencionadas? A resposta para essa pergunta é puro néctar: nós! Nós, com nossos vícios, patrocinamos a ignorância e esta ganha escala mundial.

A irresponsabilidade perante a origem da nossa alimentação, o sedentarismo, a busca pela satisfação no consumo de intoxicantes, a difusão da distração, o consumismo compulsório inserido em uma cultura de ostentação, a propagação da violência, músicas e novelas que pregam apego, raiva e inveja, a supervalorização da extroversão, a busca por relacionamentos superficiais, o narcisismo, o desequilíbrio entre o trabalho e a afetividade familiar. Todos esses são elementos que constituem a formação de uma sociedade cada vez mais caótica e desumana. E para nos livrarmos disso tudo precisamos nadar contra a maré da conveniência e executar a terefa mais difícil: encontrar em nós mesmos a origem da angústia que nos move em direção a essas fontes de desordem.

O trabalho individual necessário é o da purificação. Cada ser humano precisa libertar-se dos padrões que causam sofrimento a si próprio e ao próximo. No entanto, se levarmos uma vida distraída, nos tornamos incapazes de ver tais fontes de sofrimento em nós mesmos e simplesmente viveremos à mercê da nossa própria toxicidade, irradiando-a para todos os seres.

O caminho que vejo para que possamos nos sanar é o do autoconhecimento. Esta é a motivação para o surgimento deste livro. Os estudos propostos nesta obra não são como os conduzidos na escola industrial. Não visitaremos os assuntos para desenvolvermos uma perspectiva fria e distante de nós, como se estivéssemos tratando de algo completamente alheio, mas sim a mais íntima possível.

A dificuldade e a facilidade de se ensinar sobre espiritualidade residem nos seguintes fatos: \textit{i}) a sabedoria pregada e não praticada é hipócrita e ineficaz; \textit{ii}) a sabedoria pregada e praticada é inspiradora e benéfica. Venho, então, no direito de falar das práticas sobre as quais me sustentei ou me sustento e no dever de não excedê-las em aspecto algum.

A estrutura está disposta em quatro capítulos, os quais abordam assuntos relacionados aos âmbitos da vivência consciente típica em ordem crescente de granularidade: do mais grosseiro ao mais fino.

\begin{itemize}
\item O Corpo -- o veículo, interface para a experiência
\item O Coração -- metáfora que abrange os sentimentos
\item A Mente -- o mundo abstrato dos pensamentos
\item O Transcendental -- para além da linguagem, a Verdade
\end{itemize}

O objetivo deste recorte não é ter uma separação absoluta da Realidade em partes independentes. A postura corporal tem influência no comportamento da Mente e os pensamentos podem acionar certas emoções. Ou ainda, as emoções alimentam pensamentos que podem induzir mudanças no estado do Corpo. E o Transcendental, por sua vez, permeia toda a existência.

Este livro contém inúmeros exercícios e se não houver a disposição do leitor para praticá-los constantemente, não há necessidade de dar continuidade à leitura, visto que o auto-aperfeiçoamento espiritual é um processo de treinamento contínuo.

Será possível notar que muitas ocasiões são condizentes com mais de um exercício. Nestes casos, fica a critério do leitor escolher o exercício que melhor lhe orienta.

\chapter*{Da Perfeição do Corpo}\addcontentsline{toc}{chapter}{Da Perfeição do Corpo}

Nós podemos nos conectar com o Corpo e aprender a sentir e respeitar Seus sinais. Sujeitar-se a uma rotina desregrada e esperar que o Corpo a acompanhe é um ato de violência contra a própria natureza que resulta em uma longa série de distúrbios. Não é por acaso que os analgésicos são parte da nossa cultura.

Para a Perfeição do Corpo, elenquei quatro aspectos em ordem de e\-mer\-gên\-cia: respiração, hidratação, alimentação e postura.

Nos últimos dois séculos, nossa Medicina tem avançado bastante no entendimento da mecânica fundamental dos tecidos do Corpo. É de extrema importância buscar ajuda profissional caso seja necessário. Mas quanto à Medicina preventiva, esta nós temos a capacidade de incorporá-la em nossas vidas.

\section*{Respiração}\addcontentsline{toc}{section}{Respiração}

A respiração não só é o processo mais urgente como também uma chave valiosíssima para o auto-conhecimento. Devido à sua grandiosa importância, ela será abordada em vários outros momentos do livro.

Enquanto bebês, na maior parte do tempo a movimentação da nossa respiração é bem completa: toráxica, abdominal, lenta e ampla. Mas mesmo em recém nascidos, a respiração se torna mais curta e superficial em momentos de inquietações. À medida que crescemos, permitimos que cada vez mais elementos insalubres permeiem nossas vidas até que, adultos, não temos sequer um momento de paz por estarmos inteiramente condicionados a uma respiração paupérrima. O uso da porção inferior dos pulmões aumenta a capacidade respiratória e a exalação completa elimina o gás carbônico residual.

\begin{prat}{o máximo de vezes}
-- Sem interferir, simplesmente observe a respiração acontecer e tome consciência do seu ritmo e da sua profundidade.
\end{prat}

\section*{Hidratação}\addcontentsline{toc}{section}{Hidratação}

É uma tarefa difícil nomear uma reação química do Corpo que não ocorra em meio aquoso, portanto uma hidratação de boa qualidade é imprescindível.

\begin{prat}{o máximo de vezes}
-- Tomar consciência da frequência da sede;

-- Observar a relação entre a quantidade de água ingerida, o estado de cansaço físico, a quantidade de urina expelida e sua cor;

-- Tomar consciência das sensações de desconforto ao urinar.
\end{prat}

\section*{Alimentação}\addcontentsline{toc}{section}{Alimentação}

O ato de alimentar-se vem perdendo valor em relação às ``atividades importantes'' do dia ao ponto de tornar-se uma distração. Cada vez mais pessoas têm aderido a uma política de comer qualquer coisa para matar a fome ou pela busca da sensação prazerosa de sentir o estômago demasiadamente cheio. E para não perderem tempo enquanto comem, executam alguma outra tarefa, deixando o ato de comer em segundo plano.

A proposta aqui não é sugerir que o leitor precisa mudar a dieta, mas sim orientá-lo para sentir como tais processos estão ocorrendo no Corpo.

Há alimentos que provocam picos de prazer seguidos de baixas no humor, estimulando ciclos de dependência psicológica; que sobrecarregam o estômago e o intestino causando indisposições indesejadas e intoxicações; que acostumam o corpo a um equilíbrio não natural criando dependências químicas; que disparam processos inflamatórios em diversos tecidos etc. Enquanto que outros promovem leveza no humor, são de fácil metabolização e não causam dependências nem incômodos.

\begin{prat}{antes, durante e após cada refeição}
-- Examine se a vontade de comer provém de fato da fome;

-- Aprecie cada aspecto do ato de comer;

-- Tome consciência dos pensamentos que surgem durante a refeição;

-- Observe as sensações estomacais e intestinais após a refeição.
\end{prat}

\section*{Postura}\addcontentsline{toc}{section}{Postura}

Ser responsável pelo direcionamento da construção do Corpo é um investimento a longo prazo de extrema importância para um envelhecimento saudável. Devido ao fato do Corpo estar constantemente construindo mais de Si, nunca é tarde para começar.

Uma característica fundamental de um Corpo capaz de propiciar grande bem-estar é uma musculatura flexível, vigorosa e capaz de relaxar quando inativa, principalmente quando se trata dos grupos musculares relacionados ao sustento da coluna vertebral.

Existem diversos profissionais da área da saúde que têm competência para guiar o leitor nesta tarefa. Dentre eles, fisioterapeutas, ortopedistas, instrutores de pilates etc. Em particular, o profissional que mais recomendo é um bom professor de Yoga. Mas segue um exercício bastante poderoso e capaz de causar transformações profundas.

\begin{prat}{o máximo de vezes}
-- Corrija a coluna vertebral e abra o peitoral;

-- Percorra o corpo com o foco da atenção e relaxe cada músculo, em especial os dos ombros e da face.
\end{prat}

\chapter*{Da Perfeição do Coração}\addcontentsline{toc}{chapter}{Da Perfeição do Coração}

Vivemos em uma época na qual as dores emocionais não são devidamente respeitadas e qualquer sinal de desconforto psicológico é motivo de alarde. As pessoas estão se sentido culpadas convencidas de que estão cometendo um erro ao se sentirem tristes.

Mas sejamos sensatos aqui. Se estamos decididos a não enveredar por um caminho de violência, precisamos aprender a viver com o que sentimos.

Neste capítulo, nós daremos atenção aos nossos sentimentos e transformaremos nossas emoções em fontes de cura e de orientação.

\section*{Amadurecimento Emocional}\addcontentsline{toc}{section}{Amadurecimento Emocional}

Primeiramente, é importante que entendamos o que são as emoções na prática. Quando estamos ansiosos, podemos sentir agitações no interior do plexo solar. Quando estamos com medo, podemos sentir um frio na base do estômago. Quando estamos muito tristes, podemos sentir um desconforto na garganta. Para fins práticos, consideremos como emoções as manifestações tácteis dos sentimentos no corpo.

As duas principais formas de atrasar o amadurecimento emocional são a tentativa de levar uma vida anestesiada e o cultivo de uma atitude vitimista. No primeiro caso podemos perceber negligência e no segundo caso, teimosia.

O negligente está sujeito a cometer os mesmos erros e acumular ainda mais sofrimento latente. O teimoso incrementa a própria dor revivendo as mesmas histórias várias vezes. O fato é que precisamos ter a coragem para sentir o que a vida nos apresenta.

\begin{prat}{duas vezes por semana}
-- Sente-se com a coluna reta sobre os ísquios e descanse as mãos sobre as pernas;

-- Feche os olhos, respire conscientemente e busque a emoção no corpo;

-- Mantendo a respiração consciente, dê completa atenção ao desconforto emocional;

-- Apenas sinta dor e deixe que o seu motivo se vá.
\end{prat}

\section*{Amor Próprio}\addcontentsline{toc}{section}{Amor Próprio}

Certa vez uma pessoa me relatou a seguinte situação: ela ligava pra todos os amigos da sua lista de contatos para perguntar como estavam e para lhes desejar felicidades, mas se sentia triste porque ninguém fazia o mesmo com ela. Então pedi que ela passasse um período desejando felicidade aos seus amigos em silêncio, sem ligar para eles. Poucos dias depois ela veio falar comigo para dizer que não conseguia fazê-lo, pois sentia uma solidão imensa. Este é um ótimo exemplo para ilustrar este tópico.

Eu já sabia que ela não conseguiria executar a tarefa que lhe pedi, mas pedi mesmo assim pois ela também precisava saber que não conseguiria. Ela ligava para seus amigos buscando reconhecimento na tentativa de suprir a própria carência. Ela nem mesmo conseguia desejar-lhes felicidade verdadeiramente.

Pois bem, como podemos viver o amor próprio? Se formos devidamente sinceros, perceberemos que estamos sujeitos a diversas crenças profundamenre enraizadas. A pessoa do caso acima acreditava que precisava de confirmação externa para ser feliz, uma crença extremamente comum e que torna o exemplo ainda melhor.

Cada crença destrutiva da qual nos livramos dá mais espaço para que o amor próprio floresça em nossas vidas. Reconheço que confrontar as próprias crenças não é tarefa fácil. Afinal, temos passado décadas cultivando-as.

\begin{prat}{uma vez por semana}
-- Liste tudo que você é capaz de dizer sobre si em pequenas afirmações, principalmente as coisas que dizem respeito ao que é necessário para ser feliz;

-- Confronte cada item veementemente, mesmos os mais triviais, e destaque os que sucumbirem à sua verdade mais íntima;

-- Ao longo da semana, tome consciência de cada manifestação relacionada a uma crença destacada na lista.
\end{prat}

\section*{Compaixão}\addcontentsline{toc}{section}{Compaixão}

Um dos meus primeiros choques após dar início ao trabalho de auto-transformação foi perceber quanto sofrimento eu causei à minha volta com ações motivadas pelas minhas próprias angústias. Então me veio como um raio a compreensão de que este é um padrão humano típico: nós causamos sofrimento porque sofremos.

Reflitamos então sobre as pessoas e busquemos compreendê-las. Das mais próximas, com as quais podemos ter inimizades, às mais distantes, como políticos corruptos, todas elas estão sujeitas às mágoas da vida. A humanidade, enquanto imersa em ignorância, será um grande oceano de sofrimento.

A minha proposta é que escolhamos sair da lama do sofrimento, do mundo das inimizades. Escolhamos conscientemente a compaixão, principalmente por aqueles mais danosos pois são eles os que mais sofrem. Investiguemos o fenômeno do sofrimento em nós mesmos e nos outros.

\begin{prat}{uma vez por dia}
-- Pense em uma pessoa que lhe causa ou causou angústia (pode ser você mesmo);

-- Reconheça as causas do sofrimento desta pessoa, sem julgá-las. Se possível, uma conversa amigável é a forma mais poderosa;

-- Deseje que esta pessoa consiga se libertar do sofrimento e que alcance a paz.
\end{prat}

\chapter*{Da Perfeição da Mente}\addcontentsline{toc}{chapter}{Da Perfeição da Mente}

A Mente é uma poderosa ferramenta do artefato humano, porém a grande maioria de nós ainda é atropelada por não saber navegar em Seu espaço. A falta de fluência no domínio da Mente faz com que nós sejamos carregadas inconscientemente por pensamentos destrutivos, resultando em uma vida limitada por frustrações e ansiedades. Nós não teremos condições de prosperar verdadeiramente enquanto as orientações para lidar com a Mente não se tornarem bem difundidas no saber da humanidade.

Como alcançar a libertação mental e extinguir o ciclo da insatisfação? Como deixar de ser hipócrita ao desejar a paz mundial? Estas são perguntas que nos guiam para a Perfeição da Mente.

\section*{Os Objetos da Mente}\addcontentsline{toc}{section}{Os Objetos da Mente}

Um objeto da Mente pode se enquadrar em um dos três tipos: memória, imaginação ou julgamento.

A memória e a imaginação têm naturezas semelhantes. Normalmente são imagens, cheiros, sons ou sequências de eventos. É comum ficarmos em dúvida se algo realmente aconteceu ou se é imaginação.

Os julgamentos ocorrem por meio da voz mental. O leitor pode experimentar fechar os olhos e pensar a frase ``O planeta Terra é azul''. É como se ouvíssemos uma voz, que pode inclusive evocar um objeto da memória (ex: uma foto do planeta Terra) ou da imaginação (ex: uma esfera azul pairando no espaço).

O obstáculo para vivenciarmos a Perfeição da Mente é o fato de nos sentirmos impotentes perante o fluxo de pensamentos. São exemplos comuns de manifestações deste condicionamento a constante fuga em boas memórias, a constante fuga em prazeres imaginários, o desejo de mudar o passado, a obsessão pelo controle do futuro, a auto-flagelação e a dificuldade para perdoar.

Para que o objetivo deste livro se concretize, é imprescindível o treinamento da atitude de abrir mão voluntariamente dos objetos da Mente. Meditar não é necessariamente esvaziar a Mente, embora esta seja uma experiência posssível. Durante a meditação, nós observamos a aparição dos objetos da Mente e deixamos eles irem, tal qual acontece com todos os fenômenos da natureza: eles surgem e se extinguem.

É possível notarmos o quanto resistimos para abrir mão de certos pensamentos, mas se nos esforçamos para tal e de fato executamos esta tarefa inúmeras vezes, os benefícios são colossais. ``Mas é a minha vida! São os meus problemas!''. Este é exatamente o tipo de resistência que precisamos deixar ir, pois é possível viver em paz sem a necessidade de se identificar com pensamentos. Eu posso resolver problemas, mas os problemas não são eu. Algumas pessoas relatam que ficam mais inquietas ao meditar, mas a verdade é que estas pessoas estão apenas tomando consciência do quão inquietas elas já são.

\begin{prat}{5, 10, 15, 20, 25 ou 30 minutos por dia}
-- Delimite o tempo com o auxílio de algum sinal sonoro. Há vários aplicativos de celular capazes de desempenhar esta tarefa;

-- Sente-se com a coluna reta sobre os ísquios e descanse as mãos sobre as pernas;

-- Mantenha os olhos levemente abertos e o foco relaxado;

-- Sem interferir, observe a respiração;

-- Ao perceber a presença de qualquer pensamento, gentilmente traga atenção de volta à respiração.
\end{prat}

\section*{Presença}\addcontentsline{toc}{section}{Presença}

À medida que vamos ficando mais familiares com a prática da meditação, podemos estendê-la para o resto do dia. Começamos com atividades simples como caminhar, varrer o chão ou lavar os pratos. Então avançamos para atividades mais complexas, como as do trabalho, até conseguirmos praticar enquanto nos relacionarmos com outras pessoas, durante conversas olho a olho ou mesmo em momentos íntimos.

A frustração de muitas pessoas é tentar encarar o contato humano como uma chance de impressionar o outro e por isso expõem seus feitos e seus sonhos na busca por aprovação. Mas o que há de mais alto a ser oferecido por nós é a nossa completa presença e atenção.

Uma das grandes belezas do estado de presença é sua neutralidade com relação aos estados da mente. Aquele que tem a prática diligente não fica preso em pensamentos negativos e nem se desgasta buscando pensamentos positivos. Apenas relaxa no equilíbrio.

O estado de Presença também é uma solução para a paciência. Existem dois tipos de pessoas impacientes: as que se reconhecem impaciententes e as que parecem pacientes. Mas como é possível sofrer de impaciência estando completamente imerso no momento presente? O paciente não é aquele que aguenta as inquietações da impaciência, mas sim aquele que não as sente.

\begin{prat}{o máximo de vezes}
-- Realize atividades respirando conscientemente e com completa a\-ten\-ção ao ato de estar realizando-as;

-- Se não há atividade a fazer, realize o nada a fazer com completa atenção.
\end{prat}

\section*{Contentamento}\addcontentsline{toc}{section}{Contentamento}

Um condicionamento comum é o de postergarmos a felicidade, con\-fun\-din\-do\--a com alguma conquista/satisfação específica. E quando finalmente conseguimos o que queríamos, sentimos um grande vazio insaciável e percebemos que ainda estamos sujeitos ao mesmo padrão.

Algumas pessoas têm a ``sorte'' de adquirir uma doença, ou uma série de doenças sérias o suficiente para fazê-las repensar os próprios valores, convidando-as a dar a devida atenção às riquezas presentes em cada momento da Vida. Mas também para a nossa sorte, não precisamos adoecer seriamente. Basta que tenhamos a coragem para tomarmos consciência destes ciclos de insatisfação que tem desgastado seriamente o nosso planeta.

Ser feliz neste instante pode parecer contra-intuitivo à primeira vista, mas o fato é que é impossível ser feliz em outro momento. Em verdade, este instante místico é tudo que temos.

\begin{prat}{o máximo de vezes}
-- Mesmo que por alguns segundos, dê a si o direito de não precisar pensar, falar nem deslocar-se e apenas contemple o que está ao seu redor.
\end{prat}

\chapter*{Da Perfeição Transcendental}\addcontentsline{toc}{chapter}{Da Perfeição Transcendental}

Quando era criança, eu costumava entrar em crises existenciais porque eu não conseguia compreender a natureza do que chamava de ``eu''. Já desde muito cedo essa angústia aparentemente irremediável me abatia e eu sentia como se estivesse adentrando na minha própria morte. O que eu não sabia era que eu estava na direção correta.

Já adianto que é impossível falar \textit{a} Verdade. Nós só podemos falar \textit{da} Verdade. E o papel deste capítulo é apontar para Ela.

Outro ponto importante é que uma compreensão apenas intelectual da Verdade não é de fato o conhecimento último da Verdade. A Verdade não é apenas um objeto da Mente.

Comecemos então por imaginar que o rim aprenda a pensar e que diz ``eu'' para si mesmo. O rim pode começar a ficar incomodado com o fato do coração bombear-lhe sangue a ser filtrado ininterruptamente. O rim pode também encontrar motivos para ser feliz, grato por cada dia vivo. Mas independentemente do que o rim pense sobre si e sobre o coração, ele é parte de um organismo maior.

Já pensou se a mão esquerda começasse a se sentir inferior à direita ou se a mão direita começasse a sentir orgulho de si?

Podemos então indagar sobre quem somos nós com mais propriedade.

\begin{prat}{uma vez por dia}

-- Pergunte-se e contemple:

$\bullet$ O que é \textit{isto} que sente por esta pele?

$\bullet$ O que é \textit{isto} que observa os objetos da mente?

$\bullet$ O que é \textit{isto} que vê por estes olhos e lê \textit{estas palavras} exatamente agora?
\end{prat}

Traduzindo pobremente o Silêncio, \textit{isto} é exatamente o Universo inteiro, que vê por cada olho e escuta por cada ouvido. Quando ``eu'' escuto ``algo'', o Universo escuta a Si próprio. Quando ``eu'' cheiro ``algo'', o Universo cheira a Si próprio. Neste exato momento, o Universo lê a si próprio! Este é o máximo de Verdade que posso dizer. Se mais que isto me for requisitado, sentarei em meditação e morrerei por uns instantes.

Aquele que tem uma compreensão intelectual da Verdade está sendo chamado para o Trabalho. Aquele que teve uma experiência mística da Verdade está apenas iniciado no Trabalho. O Trabalho em sua plena execução é a Verdade incorporada: a União entre o Absoluto e o individual.

Os seres espiritualmente realizados não raciocinam em termos de certo e errado para guiarem suas vidas, pois eles sentem a Unidade em suas entranhas. Para eles, não existem mais ``minha felicidade'' nem ``seu sofrimento''. É por isso que este livro não aborda a moralidade e a retidão. Assim o fiz porque, na minha própria experiência e compreensão, a vida reta é uma consequência trivial da concentração. Quem vem desenvolvendo a auto-observação diligente sente nas vísceras o incômodo proveniente da mentira, do ódio, da exaltação etc, até que eventualmente decida por vontade própria abandonar tais costumes.

À medida que praticamos os exercícios aqui propostos, vamos descobrindo a graça de viver gentilmente. Passamos a entender o valor das experiências mais finas e sentimos cada vez menos necessidade de experiências grosseiras. Tomar uma xícara de chá pode ser uma experiência mística!

Feliz é aquele que é capaz de adentrar na Paz e deixá-La agir em sua vida.

\end{document}