Os cinco fundamentos são o alicerce para os ensinamentos. Eles não podem ser ignorados, adulterados nem encarados sem a devida importância. Em verdade, os cinco fundamentos são \emph{componentes absolutamente essenciais} do ensinamento que se aplicam tanto antes do despertar quanto, se não ainda mais fortemente, depois deste. Não se iluda pensando que os cinco fundamentos são insignificantes ou rudimentares simplesmente porque eles aparentam harmonizar-se apenas com o aspecto humano, ou relativo, da Realidade. Os cinco fundamentos são um meio de \emph{viver} e de \emph{manifestar} a natureza última da Realidade no dia a dia. Se nós não vivermos e manifestamos nas nossas vidas o que nós realizamos nos nossos momentos mais profundos de revelação, então estaremos vivendo vidas divididas.

Além disso, os cinco fundamentos proporcionam o contexto no qual os ensinamentos se desdobram. Se você remover o contexto do ensinamento, você estará removendo as salvaguardas anti-egóicas que protegem o ensinamento de interpretações autocentradas. A má interpretação de um ensinamento espiritual pelo ego é sempre um perigo significativo, pois a tendência do ego é justificar qualquer ponto de vista no qual ele investiu e ao qual se apegou.

Como adição a esse perigo, qualquer ensinamento espiritual enraizado na natureza absoluta da Realidade é, por definição, orientado em direção à Verdade, não em direção às dimensões éticas e morais da vida. Isto não significa que tais ensinamentos sejam imorais, mas sim trans-morais; ou seja, arraigados em uma realidade para além da moral e da ética relativas aos padrões da perspectiva dualística.

Não é que a moralidade seja irrelevante na visão absoluta; esta é uma má interpretação bastante comum. Significa que a moralidade não é mais embasada em valores culturais e religiosos designados para controlar os impulsos egóicos. Ao invés disso, amor autruista e compaixão fluem naturalmente da visão unificada da Realidade como expressão espontânea da Unidade. Quando nada é visto como separado ou como algo que não você, as ações que fluem de você são um reflexo da perspectiva unificada.

Pode ser que haja algumas complicações porque é possível ter \emph{alguma} experiência da Realidade última enquanto ainda não houve a libertação completa da delusão egóica. Isto acontece devido à possível mistura volátil de Realidade e ilusão simultaneamente expressando-se e existindo de uma forma inconsciente e distorcida. Enquanto que alguns casos como este podem ser esperados à medida que amadurecemos em espírito, existem algumas coisas mais distorcidas ou perigosas do que um ego que pensa ser Deus.

Muitos anos trabalhando com milhares de pessoas me mostraram que se estes aspectos fundamentais da vida espiritual forem ignorados, é muito provável que as revelações espirituais serão descarrilhadas de alguma forma. Uma falha ao explorar e entender claramente qualquer um destes, assim como ao aplicá-los cosistentemente, resultará em alguma forma de conflito, divisão e confusão.

Os cinco fundamentos são meios de recolher todos seus recursos internos --- corpo, mente e espírito --- e focá-los de forma unificada em prol da sua mais alta aspiração. Eu não tenho como enfatizar suficientemente a importância de ter um foco claro e unificado, um coração sincero e um desejo inabalável para não deludir a si e aos outros.

\section*{Clarifique sua aspiração}

Clarificar sua aspiração significa saber exatamente a motivação da sua vida espiritual, \emph{não como um objetivo para o futuro mas para cada instante}. Em outras palavras, o que você mais valoriza na vida --- não no sentido de valores morais, mas no sentido do que é mais importante para você. Contemple este questionamento. Não simplesmente assuma que você já sabe sua mais alta aspiração ou o que é mais importante para você. Vá fundo e reflita sobre o que é mais importante na sua busca espiritual; não permita que ninguém defina isto para você. Não descanse até que fique bem claro para você qual a sua aspiração.

A importância deste primeiro fundamento deve ser bem enfatizada, pois a vida se desdobra com base no que você mais valoriza. Pouquíssimas pessoas têm a Verdade ou a Realidade como valores profundos. Muitas pensam que valorizam a Verdade, mas suas ações não condizem. Em geral, as pessoas têm valores competitivos e conflitantes, o que se manifesta em conflitos internos e externos. Logo, o fato de você \emph{pensar} que algo é seu mais profundo valor não quer dizer que ele de fato o é. Ao contemplar e clarificar o que você valoriza e aspira, você se torna mais unificado, claro e certo da sua direção.

À medida que a sua realização e maturidade espiritual se aprofundam, você perceberá que alguns aspectos da sua vida permanecem firmes enquanto outros evoluem para que passem a refletir o que é relevante para o seu nível de discernimento. Desta forma criam-se maiores foco e conexão com o andamento da sua própria realização.

\section*{Consistência incondicional}

Clarificar sua aspiração é o primeiro passo. Assim, energia e atenção são conciliadas em uma força unificada que lhe move em direção à sua aspiração. Uma vez clarificada a sua aspiração, você deve seguir em direção a ela. Tal atitude diz respeito ao que você está \emph{disposto a fazer} e principalmente ao que você está \emph{disposto a deixar de fazer}.

A espiritualidade não requer que você trabalhe duro para atingir algum resultado futuro além do quanto ela requer que você esteja plenamente presente, sincero e comprometido \emph{agora}, com honestidade absoluta e disposição para revelar e abrir mão de \emph{qualquer} ilusão que esteja entre você e a realização da Verdade. Desta forma, espiritualidade não tem a ver com tempo ou com o que pode ser atingido com o tempo. Espiritualidade tem a ver \emph{apenas e sempre} com o eterno presente.

\emph{Aspiração} é mais uma questão do coração do que da mente, pois trata-se do que você mais aprecia, ama e valoriza. Você não precisa ser relembrado do que você ama, mas sim do que você \emph{não} ama. E o que você \emph{realmente} ama é mais verdadeiramente refletido nas suas \emph{ações}, não no que você sente, pensa ou fala.

Quando a aspiração alinha-se com a consistência incondicional e com o amor, ela se torna uma força muito poderosa no universo. Só assim estaremos unificados e focados o suficiente para que nossa aspiração sobreviva às eventuais dificuldades circunstanciais.

\section*{Nunca abdique da sua autoridade}

O terceiro fundamento é \emph{nunca abdicar da sua autoridade}. Isto significa que \emph{você será inteiramente responsável pela sua vida e nunca cederá isto a pessoa alguma}. No que diz respeito ao processo de iluminação em si, não faz sentido querer pegar carona nas costas de outro ser iluminado. Qualquer falha ao entender este fundamento pode levar (como tem levado em tantos casos) ao culto do fanatismo, do fundamentalismo, do pensamento místico, ao desapontamento, à desilusão e/ou à imaturidade espiritual.

É compreensível que muitas pessoas projetem seus problemas familiares, relacionamentais, sexuais, assim como problemas com Deus nos seus professores espirituais (e são muitas vezes encorajadas a assim fazerem por professores inescrupulosos), mas é essencial compreender que o papel de um professor espiritual é ser um bom e sábio guia espiritual, uma corporificação da Verdade para a qual ele(a) aponta. Pode até haver profundo respeito, amor, ou até mesmo devoção ao professor espiritual mas é fundamental não abdicar da sua própria autoridade em função da dele ou projetar toda a divindade exclusivamente para ele. Sua vida pertence às suas mãos, não às de outra pessoa. Seja devidamente responsável.

Existe uma linha muito tênue entre estar verdadeiramente aberto à orientação de um professor espiritual e regredir a uma relação infantil na qual você abdica da sua autoridade e projeta \emph{toda} sabedoria e divindade no professor. Todas as pessoas precisam encontrar um equilíbrio maduro na abertura à orientação espiritual sem abdicar da própria autoridade.

O mesmo pode ser aplicado a um ensinamento espiritual. Todo ensinamento espiritual é um dedo que aponta em direção à Realidade, nunca a Realidade em si. Estar em uma relação madura com um ensinamento espiritual requer que você o \emph{aplique}, não simplesmente acredite nele. Crenças levam a várias formas de fundamentalismo e silencia a curiosidade e investigação que são essenciais para dar espaço ao despertar e ao que há além do despertar. Um bom ensinamento espiritual é algo que se põe em \emph{prática}, é algo com o que você \emph{trabalha}. Ao fazer isto, ele trabalha em você (usualmente de forma oculta) e lhe ajuda a revelar a Verdade (e as falsidades) que adormece em você.

O que seria não abdicar da sua autoridade e ainda assim não aclamar uma falsa ou auto-centrada autoridade que lhe levará à delusão? Temo que eu não possa dizer. Ninguém pode lhe dizer como não enganar a si mesmo. Se no que há de mais profundo em você existir o desejo pela Verdade acima de tudo, mesmo que você se perca de milhares de formas, você sempre encontrará a si mesmo novamente, sendo trazido de volta para o que é Verdadeiro.

E se você não deseja a Verdade acima de tudo, bom, você já sabe para onde isso lhe levará.

\section*{Pratique a sinceridade absoluta}

Ter sinceridade autêntica é absolutamente necessário na vida espiritual. Sinceridade engloba as qualidades da honestidade, genuidade e integridade. Ser sincero não significa ser perfeito. Na verdade, o próprio esforço para ser perfeito não é sincero, pois é mais uma forma de evitar ver a si próprio como você é exatamente agora. Ser capaz e desejar ver a si próprio como você realmente é, com todas as suas imperfeições e ilusões, requer sinceridade genuina e coragem. Se estamos constantemente nos escondendo de nós mesmos, nós nunca seremos capazes de acordar da ilusão do ego.

Para que possamos ser sinceros, nós precisamos parar de julgar a nós mesmos. O julgamento encobre o acesso à verdadeira sinceridade e pode até se camuflar \emph{como} sinceridade. A verdadeira sinceridade revela uma poderosa forma de clareza e discernimento que é necessária para que possamos nos enxergar honestamente sem fuga e sem o aprisionamento dos julgamentos condicionados da mente e de seus mecanismos de defesa.

A capacidade e a vontade de ser honesto consigo é seu maior escudo contra a autossabotagem e lhe alinha com sua verdadeira aspiração. Não há maior desafio para um ser humano do que o de ser completamente honesto consigo assim como com os outros. Tal honestidade é absolutamente necessária se nós realmente desejamos acordar do sonho da separatividade e viver uma vida autêntica e unificada.

\section*{Administre bem a sua vida}

Administrar bem a própria vida significa não usar a espiritualidade para evitar aspecto algum de sua vida. Eu tenho observado que é muito comum entre pessoas que estão envolvidas com a espiritualidade inconscientemente usá-la para evitar aspectos dolorosos, problemáticos, disfuncionais, ou temerosos delas próprias. Há frequentemente uma esperança de que se eles simplesmente acordarem para a Realidade, todos os desafios irão desaparecer. Embora seja verdade que com o surgimento do despertar muitos dos aspectos que consideramos problemas vão simplesmente desaparecer, seria errado assumir que o vislumbre do estado desperto vai automaticamente resolver todos os aspectos desafiadores da vida humana.

Usar espiritualidade para evitar aspectos desafiadores de você mesmo ou do seu dia-a-dia pode inibir o surgimento da iluminação espiritual em uma grande extensão e certamente inibirá sua profundidade e estabilidade. O Caminho para a Libertação é uma forma de encarar a si e a sua vida \emph{completamente} sem fugir para negação, para os julgamento ou para pensamentos mágicos. É uma forma de cortar os véus da ilusão e despertar para a Verdade.

Ser um bom administrador da vida requer que você abrace cada aspecto da sua vida, interior e exterior, agradável ou desagradável. Você não necessariamente precisa encará-los todos de uma vez, somente aquilo que aparecer no momento. Dê a cada momento a atenção, sinceridade e comprometimento que ele merece. Falhar em tal coisa é muito mais custoso do que você jamais pode imaginar. 

Sua vida, \emph{toda} a sua vida, é o seu caminho para o despertar. Ao resistir ou não lidar com seus desafios, você irá se manter dormente à Realidade. Preste atenção no que a vida está tentando lhe revelar. Diga sim para a sua força, crueza e graça amorosa.
