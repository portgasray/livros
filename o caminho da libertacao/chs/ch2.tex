É muito fácil perder tempo e energia quando a busca espiritual é orientada para coisas que tem pouca ou nenhuma relevância para o despertar. As Três Ideias Norteadoras provém o arcabouço conceitual no qual o ensinamento repousa e orienta a mente para os princípios chaves que levam à iluminação. Essas Ideias Norteadoras dão foco e direção para as Três Práticas Centrais que são descritas mais adiante neste livro.

\section*{A Questão de Ser}

Em cima da entrada do Oráculo de Delphi estava escrito as palavras ``Conheça a Ti Mesmo''. Jesus veio e adicionou um senso de urgência e consequência para a idéia anciã quando ele falou: ``Se você traz para fora o que está dentro de você, o que você traz para fora lhe salva. Se você não traz pra fora o que está dentro de você, o que você não traz para fora lhe destrói.''

O que Jesus está dizendo é que espiritualidade é algo sério e com consequências sérias. Sua vida se mantém precariamente no equilíbrio, balançando entre estados de sonambulismo inconsciente e de plena atenção. O fato que a maioria das pessoas não percebem suas vidas assim testifica o quão profundamente dormentes e em negação elas ainda estão.

Então oque é istoque precisamos trazer para fora?

Dentro de cada uma de nossas formas repousa um mistério existencial do \emph{ser}. Além da aparência física, personalidade, gênero, história, ocupação, esperanças e sonhos, ires e vires, ali jaz um misterioso silêncio, um abismo de quietude carregada de uma presença etérea. Mesmo com todos os nossos negócios ansiosos e obcessões por trivialidades, nós não conseguimos ignorar completamente esta essência fantasmagórica no nosso centro. E ainda assim, nós fazemos tudo que podemos para evitar a sua quietude, seu silêncio e a sua vacuidade e intimidade radiante

\emph{Ser} é aquilo que incomoda a nossa insistência em permanecer no reino do entorpecimento do nosso desespero secreto. É a coceira que não pode ser coçada, o assobio que não pode ser ignorado. Ser, realmente ser, não é apenas uma crença.

A maioria de nós vive em um estado no qual o nosso \emph{ser} foi há muito exilado para o reino sombrio da angústia silenciosa. Certas vezes, o \emph{ser} romperá o tecido da nossa inconsciência para nos lembrar que não estamos vivendo a vida que poderíamos estar vivendo, a vida que realmente importa. Em outros momentos, o \emph{ser} vai retroceder para os bastidores, silenciosamente aguardando a nossa devota atenção. Mas não se engane: \emph{ser} --- o seu \emph{ser} --- é o assunto central da vida. 

Permanecer inconsciente do \emph{ser} é estar preso em uma terra desolada e conflituosa governada pelo ego, que parece normal porque nos deixamos levar para um estado de ignorância no qual uma larga camada de ódio, desonestidade e ganância é vistas como normail e sã. Mas na verdade, nada poderia ser menos são do que o que nós seres humanos temos chamado de realidade. 

Ao nos apegarmos ao que sabemos e acreditamos, nos prendemos ao movimento da nossa imaginação e pensamento condicionado, ao mesmo tempo que acreditamos ser perfeitamente racionais e sãos. Portanto nós continuamos a fortificar a realidade do que causa tanta dor e sofrimento a nós mesmos e aos outros.

No fundo, todos nós suspeitamos que algo está muito errado com a forma que percebemos a vida, mas nós fazemos de tudo para correr deste fato. E a forma pela qual nós permanecemos cegos para a nossa aterrorizante condição através de uma negação obsessiva e patológica do \emph{ser} --- como se algo terrível fosse acontecer conosco se encararmos a pura luz da Verdade e enterrarmos nossos apegos à ilusão. 

É dentro da dimensão do \emph{ser} que a Verdade se revela -- não a verdade da matemática ou química, filosofia ou história, mas a Verdade que começa a se desdobrar naqueles momentos de quietude nos quais a rotina comum da vida de repente se torna transparente para um senso de sentido e significância desconhecido na maior parte do tempo. Estes encontros vitais e inesperados com o \emph{ser} indicam a Verdade que jaz logo abaixo do tecido da nossa vida cotidiana, relembrando-nos que a vida que nos apegamos pode ser mais boba do que jamais imaginamos e que há uma Realidade que tem o poder de desbloquear o mistério das nossas vidas se nós decidirmos abandonar nossos compromissos com o medo e a segurança que tanto conhecemos.

Todos nascemos com o \emph{ser} velado em obscuridade. Nós podemos reconhecer a transparência do \emph{ser} brilhando nos olhos de um infante, mas tal \emph{ser} não é consciente de si mesmo. Os bebês vivem em um mundo mágico do \emph{ser} inconsciente, enquanto que os adultos vivem em um mundo de separação egocêntrica e de negação do \emph{ser}. O despertar espiritual torna possível a restauração e a retificação do \emph{ser} para o seu domínio verdadeiro. 

A questão de \emph{ser} é tudo. Nada poderia ser mais importante ou consequente --- nada onde os riscos correm tão alto. Permanecer inconsciente de \emph{ser} é permanecer dormente para a nossa própria realidade e portanto dormente à Realidade em larga escala. A escolha é simples: despertar para o \emph{ser} ou dormir um sono sem fim.
