\emph{O Caminho da Libertação} é um guia prático e sucinto para a libertação espiritual, às vezes chamada despertar, iluminação, autorrealização ou simplesmente enxergar o que é absolutamente Verdadeiro. É impossível saber o que palavras como \emph{libertação} ou \emph{iluminação} significam até que você as realize por si mesmo. Assim, é inútil especular sobre o que a iluminação é; em verdade, fazê-lo é um grande obstáculo para que ela aconteça. Um princípio orientador é que perceber progressivamente o que não é absolutamente Verdadeiro é infinitamente mais valoroso do que especular sobre o que é.

Muitas pessoas pensam que a função de um ensinamento espiritual é prover respostas para os grandes questionamentos da vida, mas na verdade o oposto é verdadeiro. A principal tarefa de um bom ensinamento espiritual não é responder seus questionamentos, mas questionar suas respostas. Pois são suas premissas conscientes e inconscientes e suas crenças que lhe fazem ver divisão onde existe apenas unidade e completude.

A Realidade que para a qual estes ensinamentos apontam não está escondida, não é secreta ou distante. Você não pode ganhá-la, merecê-la ou compreendê-la. Exatamente neste instante, Realidade e completude estão à vista. De fato, a única coisa que existe para ver, ouvir, cheirar, provar, tocar ou sentir é Realidade, ou Deus se você preferir. a completude absoluta lhe cerca onde quer que você esteja. Logo, não há motivo pra se preocupar com isto, a não ser pelo fato de que nós, humanos, há muito tempo nos colocamos em um emaranhado de confusão e desordem tão obscuro que dificilmente consideraríamos a existência, e ainda mais dificilmente experienciaríamos por nós mesmos, a divindade dentro de nós e ao nosso redor.

\emph{O Caminho da Libertação} é um chamado à ação; é algo que você \emph{faz}. É um \emph{fazer que lhe desfará completamente}. Se você não \emph{praticar} o ensinamento, se você não estudá-lo e aplicá-lo destemidamente, nenhuma transformação ocorrerá. \emph{O Caminho da Libertação} não é um sistema de crenças. É inteiramente prático.

Ler este livro como um espectador seria perder o ponto fundamental. Ser um espectador é fácil e seguro; ser um participante ativo no seu próprio despertar para a Verdade não é nem fácil nem seguro. O caminho adiante é imprevisível, o compromisso é absoluto e os resultados não são garantidos. Você realmente achou que haveria outra forma?

Se você comparar \emph{O Caminho da Libertação} a outros ensinamentos, ou interpretá-lo através de lentes de outros ensinamentos, inevitavelmente você irá cometer interpretações errôneas. Nos dias de hoje, com acesso instantâneo aos ensinamentos espirituais, este é um problema especialmente pervasivo. As pessoas interpretam erroneamente as coisas que eu digo porque elas estão filtrando sob as lentes de outros ensinamentos espirituais que usam vocabulários semelhantes. Logo, eu sugiro que você aborde estes ensinamentos sem filtrá-los com entendimentos prévios da mente.

Nenhum ensinamento espiritual é um caminho direto para a iluminação. Na verdade, não há caminho para a iluminação, simplesmente porque iluminação sempre está presente em todos os locais em todos os momentos. O que você \emph{pode fazer} é remover todas as ilusões, especialmente as que você mais valoriza e nas quais sente segurança, que obscurecem sua percepção da Realidade. Abandone suas ilusões e suas resistências, e a Realidade subitamente se tornará visível.

\emph{O Caminho da Libertação} é uma medicina para curar várias doenças espirituais. Assim como a medicina em si não é boa para a saúde, mas um meio para chegar a ela, estes ensinamentos não são a Verdade, mas um meio para realizá-la. O sábio indiano Ramana Maharshi comparou os ensinamentos espirituais a espinhos usados para remover outros espinhos. E pessoalmente eu gosto desta analogia.

Estudar \emph{O Caminho da Libertação} é estudar a si mesmo. Estudar a si mesmo não significa adicionar mais conhecimento ao seu cérebro com ideias sobre si mesmo, mas remover todos os conceitos e características com os quais você habitualmente identifica o seu eu: nome, raça, gênero, ocupação, status social, passado, assim também como os julgamentos psicológicos que você faz sobre si. Quando o eu é reduzido ao seu cerne essencial, tudo que pode ser dito é \emph{eu sou; eu existo}.

O que, então, é este \emph{eu} que existe?

Este não é um livro sobre melhorias, auto-aperfeiçoamento ou estados alterados de consciência. É sobre o despertar espiritual, indo do estado de sonho do ego ao o estado desperto para além do ego o mais rápido e eficientemente possível. A jornada não é como as pessoas antecipam e iluminação não é como é vendida normalmente. Não lhe ensinarei a atingir êxtase ou felicidade infinita, a encontrar sua alma gêmea, ou dez passos para se tornar milionário. Eu não acredito em propaganda enganosa nem em falsas promessas. Muitos buscadores espirituais já vivem sob uma constante dieta de ensinamentos sem qualidade, aquelas platitudes bonitas de se escutar que causam pouca ou nenhuma transformação além de aliviar insatisfações inerentes ao estado de sonho. Se você gosta deste tipo de coisa, este livro não é para você.

Eu deixei pistas para a realização da Verdade ao longo deste livro, da primeira à última página. Não assuma que os elementos mais importantes destes ensinamentos são fáceis de serem percebidos ou claramente enfatizados. Eles são costurados neste livro assim como fios são transformados em tecido, fáceis de passarem despercebidos se você não tiver os olhos para vê-los ou a sinceridade para compreendê-los. Não é que eu queira ser obscuro --- eu faço tudo que eu posso para \emph{não} ser obscuro --- mas a Verdade não é algo que possa ser profunda e completamente entendida se lhe for dada de forma mastigada. Tais verdades são como \emph{fast food}, de fácil acesso mas não satisfatórias ao longo prazo.

Na nossa sociedade moderna nós esperamos que tudo nos seja dado em porções fáceis de serem consumidas, de preferência muito rapidamente para que possamos continuar com nossas vidas apressadas. Mas a Verdade não se colocará em conformidade com nossas formas frenéticas de evitar a Realidade ou com nossos desejos de ter o todo de algo pelo menor investimento de tempo e energia.

Você tirará dos ensinamentos do \emph{Caminho da Libertação} exatamente o que você colocar neles. Estes ensinamentos precisam ser estudados, contemplados e colocados em prática, não simplesmente lidos por entretenimento. Um sábio uma vez disse, ``A prova de um desejo é encontrada no hábito de resposta.''

Também é necessário compreender que \emph{O Caminho da Libertação} não é nem uma forma de psicoterapia nem uma solução para todos os desafios que os seres humanos enfrentam em seus dias. Embora aplicações terapêuticas possam ser necessárias e úteis para algumas pessoas, elas não são o foco destes ensinamentos.

O despertar não é nem uma cura para tudo que lhe aflinge, nem uma escapatória das dificuldades da vida. Tal pensamento mágico vai na direção oposta ao desdobramento da Realidade e é um grande impedimento para que esta se expresse da forma mais madura. O objetivo destes ensinamentos é acordar para a natureza absoluta da Realidade e então encorporá-la e vivê-la em sua máxima extensão. Tal despertar eventualmente pode trazer um sentimento de paz profunda, amor, bem-estar, mas estes são apenas subprodutos do estado desperto, não o objetivo.

Não é a busca por estados de felicidade e bem-aventurança cada vez mais elevados que levam à iluminação, mas o desejo ardente pela Realidade e a sincera insatisfação com uma vida que não seja inteiramente autêntica.

\section*{Acorde ou pereça}

Os problemas do mundo são, em geral, problemas humanos --- a consequência inevitável do sonambulismo egóico. Se olharmos com cuidado, todos os sinais estão presentes para sugerir que não somos apenas sonâmbulos, mas chegamos também aos limites da insanidade. De certa forma, perdemos (ou pelo menos esquecemos) nosso espírito, e nós tentamos arduamente ignorar este fato pois não queremos ver o quão dormente estamos, o quão desolada nossa condição realmente é. Então nós seguimos cegamente, movidos por forças que não reconhecemos nem entendemos, ou nem mesmo percebemos.

Sem dúvidas nós estamos em um momento muito crítico. Nosso mundo se sustenta em um equilíbrio muito precário. Despertar para a Realidade não é mais apenas uma possibilidade, mas uma necessidade. Nós navegamos no navio da delusão para tão longe quanto poderíamos ir. Nos jogamos ao mar e agora nos encontramos encalhados em uma ilha desolada. Nossas opções implodiram. ``Acorde ou pereça'' é o chamado espiritual dos nossos tempos. Será que vamos precisar de mais motivação do que isto?

E mesmo assim, tudo está eternamente bem, e melhor do que pode ser imaginado.
