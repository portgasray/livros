\chapter[(Ashtavakra)]{Aprisionamento e Libertação (Ashtavakra)}

[8.1] Aprisionamento é quando a mente deseja ou se aflinge, aceita ou rejeita, é satisfeita ou insatisfeita por conta de coisas.

[8.2] Quando a mente nem deseja nem se aflinge, nem aceita nem rejeita, nem é satisfeita nem insatisfeita por conta de coisas, a libertação está ao alcance.

[8.3] Quando a mente está apegada a alguma experiênca, isto é aprisionamento. Quando a mente não está apegada a experiência alguma, isto é libertação.

[8.4] Quando não existe um ``eu'', há apenas a libertação. Quando o ``eu'' aparece, surge também o aprisionamento. Compreendendo isto, não aceitar nem rejeitar se tornam práticas sem esforço.
